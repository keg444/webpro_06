\documentclass{jsarticle}

\usepackage[dvipdfmx]{graphicx}
\usepackage{tabularx}
\usepackage{booktabs}
\usepackage[dvipdfmx]{xcolor}
\usepackage{listings} 

\title{Webプログラミングレポート2 仕様書}
\author{24G1051 久峩 丈}
\date{}

\begin{document}
\maketitle
\section*{実施内容}
BBSに3つ以上の機能を追加する

\section*{利用者向け}

\subsection*{手順}
\begin{enumerate}
\item ユーザー名とメッセージを入力し,投稿ボタンを押す.
\item 返信したいときはReplyボタンを押し返信内容を入力し,投稿ボタンを押して返信する.
\item 投稿チェックボタンで今までの投稿を一覧表示
\end{enumerate}


\section*{管理者向け}
\begin{enumerate}
\item ターミナルで$\ldots$/webpro\_06/に移動し node app8.js でサーバを起動する.
\item Example app listening on port 8080!と表示されたら起動成功.
\item ブラウザでhttp://localhost:8080/public/bbs.htmlにアクセスし,掲示板(BBS)を起動する.
\end{enumerate}

\section*{開発者向け}
\subsection*{追加した機能}
\begin{itemize}
    \item 投稿順整理
    \item 投稿と同時に投稿を一覧表示
    \item リプライ
\end{itemize}

\subsection*{投稿チェック}
\subsubsection*{関数名}
checkPosts
\subsubsection*{引数}
なし
\subsubsection*{機能}
入力した文字を掲示板に投稿する
\subsubsection*{処理}
\begin{enumerate}
    \item サーバーが返す投稿数とクライアントの保持投稿数が一致するか確認し,一致したら終了.
    \item POSTメソッドでURL"/read"にリクエスト送信し,新しい投稿を取得する.
    \item リプライ用のボタンとテキストボックスを配置する.テキストボックスには\>@{投稿者名}がデフォルトで入力される.
    \item 入力された名前,メッセージ,投稿順とリプライを表示し,投稿数を更新.
    \item 処理が成功したら[ 'Name1', 'Message1' ] read $->$ 0 のようにターミナルに出力される.
    $\cdot$ 投稿するとき(document.querySelector('\#post').addEventListener('click', () =\>  以下)にcheckPostsを呼び出すことで投稿と同時に投稿の一覧を表示する.
\end{enumerate}

\subsection*{リプライ投稿}
\subsubsection*{関数名}
replyToPost
\subsubsection*{引数}
postId(リプライ対象),replyMessage(入力したメッセージ),replyElement(リプライ投稿の本体)
\subsubsection*{機能}
ユーザの投稿に対してリプライ(返信)メッセージを送信する.
\subsubsection*{処理}
\begin{enumerate}
    \item postIdとreplyMessegeをコンソールに出力する.
    \item POSTメソッドでURL"/reply"にリクエストを送信する.
    \item レスポンスがtrueの場合リプライを表示し,falseの場合エラーを出力.
\end{enumerate}


\end{document}  